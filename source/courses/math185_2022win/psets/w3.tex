\documentclass[10pt,psamsfonts,reqno,oneside,letterpaper]{amsart}
\usepackage[dvips,text={6.5truein,9truein},left=1truein,top=1truein]{geometry}
\usepackage{amssymb,amsmath,amscd,enumerate}
\usepackage{amscd}
\usepackage[pdftex]{graphicx}
\usepackage{graphicx}
\usepackage{caption}
\usepackage{subcaption}
\usepackage[colorlinks,linkcolor=black,citecolor=black,pdfstartview=FitH]{hyperref}
\usepackage{pgfplots}
\usetikzlibrary{calc}

% Uncomment to use syncing

%\usepackage{pdfsync}


% Paragraphs
\parindent=0pt
\parskip=5 pt plus 2 pt minus 1pt

%color shortcuts
\usepackage{color}
\newcommand\change[1]{{\color{red}#1}}
\newcommand\delete[1]{{\color{green}#1}}
\newcommand\comment[1]{{\color{blue}#1\color{black}}}

\newcommand{\xdownarrow}[1]{%
	{\left\downarrow\vbox to #1{}\right.\kern-\nulldelimiterspace}
}

\newtheorem{theorem}{Theorem}
\newtheorem{corollary}{Corollary}
\newtheorem{definition}{Definition}
\newtheorem{proposition}{Proposition}
\newtheorem{lemma}{Lemma}

%other shortcuts
\newcommand\Z{{\mathbb Z}}
\newcommand\N{{\mathbb N}}
\newcommand\C{{\mathbb C}}
\newcommand\Q{{\mathbb Q}}
\newcommand\R{{\mathbb R}}
\newcommand\id{{\mathrm{id}}}
\newcommand\im{{\mathrm{im}}}

\begin{document}
	
	
	\section*{Math 18500 - Problem Set 3}
	\begin{enumerate}[I]
	\item For each homogeneous system of linear equations:
	\begin{enumerate}[(1)]
		\item Write the system in matrix form.
		\item Determine the initial velocity of the solution with initial values $(x(0), y(0)) = (1,0)$. \\
		\textit{Do not find this solution, just determine its initial velocity based on the velocity field.}
		\item Find the general solution of the system.
		\item Roughly sketch a few solutions of the system which exhibit all possible different behaviors.  Make sure that the solution starting at $(1,0)$ has a direction which is consistent with the velocity you calculated in step (2). \\
		\textit{You can use a computer to check your work, but do the sketch by hand first.}
		\item Find the solution with initial values $(x(0),y(0)) = (1,0)$ explicitly.
	\end{enumerate}
	
	\begin{center}
		\begin{tabular}{lll}
			\textbf{a}. $\left \{ \begin{array}{l} x' = 9x - 2y \\ y' = -2x + 6y \end{array} \right. $ &
			\textbf{b}. $\left \{ \begin{array}{l} x' = - x + y  \\ y' = - 2x - 4y \end{array} \right.$ &
			\textbf{c}. $\left \{ \begin{array}{l} x' = 2x + 4y   \\ y' = x - y  \end{array} \right.$ \\ \\
			\textbf{d}. $\left \{ \begin{array}{l} x' = - x + 4y \\ y' = - 4x - y  \end{array} \right.$ &
			\textbf{e}. $\left \{ \begin{array}{l} x' = x - 2y \\ y' = x - y \end{array} \right. $ &
			\textbf{f}. $\left \{ \begin{array}{l} x' = x - y \\ y' = - x + y \end{array} \right. $ 
			%\textbf{f}. $\left \{ \begin{array}{l} x' =  -2x + 5y   \\ y' = -2x + 4y  \end{array} \right.$ 
		\end{tabular}
	\end{center}
	\textit{Hint: For $\textbf{f}$, you may use a computer if you don't know how to make a good sketch.   But once you see the solutions on the computer, try to explain why they look that way, based on the form of your solution.}
	
	\item Consider the inhomogeneous system of linear equations
	\[ \left \{ \begin{array}{rcl} \dfrac{dx}{dt} &=& 1 - x \\ \\
		\dfrac{dy}{dt} &=& x - y - 2\end{array} \right. \]
	\begin{enumerate}
		\item Write the system in matrix form, 
		\[ \frac{d \vec{x}}{dt} = A \vec{x} + \vec{b} \]
		and show that the matrix $A$ does not have an eigenbasis (real or complex).  Therefore, our standard methods will not apply to this system and we will be forced to use a different method to solve it.
		\item Find the equilibrium point of the system.
		\item Find the general solution of the corresponding homogeneous system, 
		\[ \frac{d \vec{x}_h}{dt} = A \vec{x}_h. \]
		\textit{Hint: First solve for $x_h$ using the equation which does not involve $y_h$.  Then plug the result in to the other equation, and solve for $y_h$.} 
		\item Find the general solution of the original (inhomogeneous) system.
		\item Based on your solution, is the equilibrium stable or unstable?  In other words, do the solutions tend towards the equilibrium point as $t \to \infty$, or do they tend away from it?
		\item Use the following tool to plot the solutions and see what they look like:
		\begin{center} \url{https://homepages.bluffton.edu/~nesterd/apps/slopefields.html} \end{center}
		Click around to see a number of solutions and make a sketch of what you see on the computer. 
	\end{enumerate}
	
	\item A standard technique for studying equilibrium points of nonlinear systems of ODE,
	\[ x' = P(x,y) \]
	\[ y' = Q(x,y) \]
	is to use \emph{linear approximations} of the components of the velocity field:
	\[  P(x,y) \approx P(x_0,y_0) + P_x(x_0,y_0) (x-x_0) + P_y(x_0,y_0) (y-y_0) \]
	\[  Q(x,y) \approx Q(x_0,y_0) + Q_x(x_0,y_0)  (x-x_0) + Q_y(x_0,y_0) (y-y_0). \]
	In the vicinity of an equilibrium point (i.e. a point where $P(x_0,y_0) = Q(x_0,y_0) = 0$), solutions of the \emph{linearized system} 
	\[ x' = P_x(x_0,y_0) (x-x_0) + P_y(x_0,y_0) (y-y_0)  \]
	\[ y' = Q_x(x_0,y_0)  (x-x_0) + Q_y(x_0,y_0) (y-y_0)\]
	are accurate approximations of the solutions of the nonlinear system (during any interval of time in which the solution remains close to the equilibrium).
	\begin{enumerate}
		\item Consider the nonlinear system of differential equations 
		\[ x'  = x^2 - y^2 + 3 = P(x,y) \]
		\[ y'  = 5 - x^2 - y^2  = Q(x,y). \]
		Identify all equilibrium points of the system, by setting $x' = y' = 0$. \\
		\textit{You should find a total of four equilibrium points.}
		\item Make a schematic sketch of the velocity field of the system, by carrying out the following steps:
		\begin{enumerate}
			\item Draw the \textit{nullclines}.  These are the isoclines where the velocity field has slope $0$ or $\infty$.  Equivalently, they are the curves along which one component of the velocity is zero ($P = 0$ or $Q = 0$). 
			\textit{Note: The equilibrium points should be located where the nullclines intersect.}
		\item The nullclines divide the $xy$ plane into several regions.  For each region, determine whether $P$ and $Q$ are positive or negative in the region, and draw a vector in the region which is consistent with these signs (e.g. up and to the right, if $P>0$ and $Q>0$). \\
			\textit{Tip: First evaluate the velocity field at the point $(0,0)$, and draw this vector in your picture.}
		\item Draw velocity vectors of the correct direction along the nullclines (e.g. up, if $P=0\textrm{ and }Q>0$).  The vectors you draw will have different directions along different segments of each nullcline - make sure that they are consistent with the vectors you have drawn in the neighboring regions.
		\end{enumerate}
		
		\item For each equilibrium point you found in part \textbf{a}, write down the corresponding linearized system, and convert it into matrix form. \\
		\textit{Tip: First write out general formulas for $P_x$, $P_y$, $Q_x$, and $Q_y$, then plug in the equilibrium points.}
		\item  For each linearized system you wrote down in part \textbf{c}, does it have real or complex eigenvalues? 
		\item For each linearized system you wrote down in part \textbf{c}, is it stable or unstable? \\
		\textit{Note: Because solutions of the linearized system approximate solutions of the nonlinear system, this tells you whether the corresponding equilibrium of the nonlinear system is stable or unstable.}
		\item  Use the following tool to visualize the solutions of the system:
		\begin{center} \url{https://homepages.bluffton.edu/~nesterd/apps/slopefields.html} \end{center} 
		Click around to see what solutions do when they start near each equilibrium point.  Is their long-term behavior consistent with your solution of part \textbf{e}?  Explain.  Visually speaking, what is the difference between the equilibria where the linearization has real vs. complex eigenvalues? 
	\end{enumerate}
	\end{enumerate}	
\end{document} 




